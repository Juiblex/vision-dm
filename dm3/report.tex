\documentclass[a4paper]{article}
\usepackage[utf8]{inputenc}
\usepackage[english]{babel}
\usepackage[T1]{fontenc}
\usepackage{lmodern}
\usepackage{fullpage}
\usepackage{amsmath}
\usepackage{graphicx}
\usepackage{framed}
\usepackage{listings}
\usepackage{placeins}
\usepackage{subcaption}
\usepackage{array}
\usepackage[justification=centering]{caption}

\author{Théotime Grohens}
\title{Introduction to Computer Vision: \\ Assignment 3: Mean-shift}

\begin{document}

\maketitle

\section*{Question 3}

Decreasing $r$ increases the runtime of the program, since there are more different peaks to be reached and hence more paths to be computed.
Increasing $c$ causes less points to be incorporated into existing search paths, and so increases the runtime as well.

As for the visual quality of the images, smaller values of $r$ yield more peaks and thus more different colors, whereas higher values of $r$ diminish the number of different images, and might even yield only one single peak, which completely destroys the image (as in the sheep images for $r$=20).
Changing $c$ affects the peak to which each point maps; higher values of $c$ merge less points, and make the image crisper.

\end{document}